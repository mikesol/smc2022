% -----------------------------------------------
% Template for SMC 2022
% based on SMC 2022 template
% -----------------------------------------------

\documentclass{article}
\usepackage{smc}
\usepackage{times}
\usepackage{ifpdf}
\usepackage[english]{babel}
\usepackage{cite}

%%%%%%%%%%%%%%%%%%%%%%%% Some useful packages %%%%%%%%%%%%%%%%%%%%%%%%%%%%%%%
%%%%%%%%%%%%%%%%%%%%%%%% See related documentation %%%%%%%%%%%%%%%%%%%%%%%%%%
%\usepackage{amsmath} % popular packages from Am. Math. Soc. Please use the
%\usepackage{amssymb} % related math environments (split, subequation, cases,
%\usepackage{amsfonts}% multline, etc.)
%\usepackage{bm}      % Bold Math package, defines the command \bf{}
%\usepackage{paralist}% extended list environments
%%subfig.sty is the modern replacement for subfigure.sty. However, subfig.sty
%%requires and automatically loads caption.sty which overrides class handling
%%of captions. To prevent this problem, preload caption.sty with caption=false
%\usepackage[caption=false]{caption}
%\usepackage[font=footnotesize]{subfig}


%user defined variables
\def\papertitle{Zero-cost abstractions to manage audio resource allocation in functional audio languages}
\def\firstauthor{Mike Solomon}

% adds the automatic
% Saves a lot of output space in PDF... after conversion with the distiller
% Delete if you cannot get PS fonts working on your system.

% pdf-tex settings: detect automatically if run by latex or pdflatex
\newif\ifpdf
\ifx\pdfoutput\relax
\else
   \ifcase\pdfoutput
      \pdffalse
   \else
      \pdftrue
\fi

\ifpdf % compiling with pdflatex
  \usepackage[pdftex,
    pdftitle={\papertitle},
    pdfauthor={\firstauthor},
    bookmarksnumbered, % use section numbers with bookmarks
    pdfstartview=XYZ % start with zoom=100% instead of full screen;
                     % especially useful if working with a big screen :-)
   ]{hyperref}
  %\pdfcompresslevel=9

  \usepackage[pdftex]{graphicx}
  % declare the path(s) where your graphic files are and their extensions so
  %you won't have to specify these with every instance of \includegraphics
  \graphicspath{{./figures/}}
  \DeclareGraphicsExtensions{.pdf,.jpeg,.png}

  \usepackage[figure,table]{hypcap}

\else % compiling with latex
  \usepackage[dvips,
    bookmarksnumbered, % use section numbers with bookmarks
    pdfstartview=XYZ % start with zoom=100% instead of full screen
  ]{hyperref}  % hyperrefs are active in the pdf file after conversion

  \usepackage[dvips]{epsfig,graphicx}
  % declare the path(s) where your graphic files are and their extensions so
  %you won't have to specify these with every instance of \includegraphics
  \graphicspath{{./figures/}}
  \DeclareGraphicsExtensions{.eps}

  \usepackage[figure,table]{hypcap}
\fi

%setup the hyperref package - make the links black without a surrounding frame
\hypersetup{
    colorlinks,%
    citecolor=black,%
    filecolor=black,%
    linkcolor=black,%
    urlcolor=black
}


% Title.
% ------
\title{\papertitle}

% Authors
% Please note that submissions are NOT anonymous, therefore
% authors' names have to be VISIBLE in your manuscript.
%
% Single address
% To use with only one author or several with the same address
% ---------------
\oneauthor
   {\firstauthor} {Affiliation1 \\ %
     {\tt \href{mailto:author1@smcnetwork.org}{author1@smcnetwork.org}}}



% ***************************************** the document starts here ***************
\begin{document}
%
\capstartfalse
\maketitle
\capstarttrue
%
\begin{abstract}
Audio programming languages fall into two broad categories: imperative (CSound, sclang) and functional (Faust, tidal). For the latter, the declarative and immutable nature of data and control structures are often at odds with the mutable and referentially opaque nature of audio units like oscillators and filters. Recently, languages such as Rust have simplified this task through zero-cost abstractions to manage resource allocation in otherwise functional settings. This paper applies these techniques to functional audio programming languages. In addition to correct and low-latency automatic audio memory management, we will see how these techniques allow for efficient namespacing, drastically reducing the amount of time it takes for ceratin graph traversals.
\end{abstract}
%

\section{Introduction}\label{sec:introduction}
The prevailing resource-management paradigm in all modern DAWs and most visual or text-based music creation environments is to model resources as units that are connected to each other. Several examples of this are shown in \ref{fig:audio-units}.

\begin{figure}[t]
\centering
\includegraphics[width=0.6\columnwidth]{figure}
\caption{Figure captions should be placed below the figure,
exactly like this.\label{fig:audio-units}}
\end{figure}

% web audio example
% max example
% supercollider example

Almost all audio units preserve some notion of state internally. For example:

\begin{itemize}
\item Waveforms remember their previous position in a lookup table so that changes in frequency do not disrupt phase.
\item Biquad filters and delay lines need to access a certain number of input samples in the past as well as potentially their own output.
\item FFT-based algorithms need to access and aggregate previous FFTs using a windowing function before rendering the output in the time domain.
\end{itemize}

Because of this, subbing out one audio-unit for another cannot be done in the same way that one would substitute a referrentially transparent value like a floating-point number or boolean. In the most extreme cases, for example with a convolution unit, there are undesirable audio artifacts coupled with seconds of lost rendering data.

In imperative audio programming langauges, such as the Web Audio API or PureData, managing stateful units is a core feature of the language. Variables reference a specific generator that on which arbitrary side effects, like setting a frequency or a volume, can be performed:

%

However, in graph-based langauges, this is more challenging. Consider, for example, the following pseudocode:

\being{verbatim}
render time =
  if time < 5.0
  then sinOsc (time * 220)
  else if time < 10.0
  then sinOsc (time * 50)
  else highpass (playBuf "hello")
\end{verbatim}

We are now demanding much more of the audio engine:
\begin{itemize}
\item Even though we haven't specified it, in almost all circumstances we will want to preserve the sine-wave oscillator across the five-second mark to avoid phase disruption.
\item We do not want the engine to spuriously tear down or create new oscillators, for example arbitrarily changing oscialltor units before the 10-second mark.
\item We want as little runtime accounting as possible, ideally precompiling an efficient kernel to minimize how much time is spent on our control thread.
\end{itemize}

As graph-based languages such as Tidal Cycles and \verb=purescript-wags= become more prevelant, it is important to maintain the helpful abstraction of graphs as a pure function of time while rigorously keeping track of active and inactive audio units with as little computational overhead as possible.

This paper proposes a zero-cost abstraction to manage resource allocation in functional audio programming languages combining the following three techniques:

\begin{enumerate}
  \item Cofree comonads, as popualrized by Edward Kimett in ??.
  \item Existential quantification~\cite{Someone:00, Someone:09}, first introduced as a programming paradigm by Nigel Perry https://downloads.haskell.org/~ghc/latest/docs/html/users_guide/exts/existential_quantification.html#existentially-quantified-data-constructors.
  \item Substructural typing, a system popularized by Rust's borrow checker and recently incorporated into GHC XX and other functional contexts.
\end{enumerate}

The paper will give examples of all three concepts, showing how they can be combined into an audio-resource management strategy and finally giving several real-world stress test and implementation examples using \verb=purescript-wags=.

\section{Cofree comonads}
\label{sec:cofree_comonads}

\section{Existential quantification}
\label{sec:existential_quantification}

\section{Affine types}
\label{sec:affine_types}

\section{Tying it together}
\label{sec:tying_it_together}

\section{Examples and benchmarks}
\label{sec:examples_and_benchmarks}

\section{Conclusion}
\label{sec:conclusion}


\subsection{Footnotes}
You can indicate footnotes with a number in the text \footnote{This is a footnote example.},
but try to work the content into the main text.Use 8~pt font-size for footnotes.  Place the footnotes at the bottom of the page
on which they appear. Precede the footnote with a 0.5~pt horizontal rule.

\section{Citations}
All bibliographical references should be listed at the end, inside a section named ``REFERENCES''. References must be numbered in order of appearance. You should avoid listing references that do not appear in the text.

Reference numbers in the text should appear within square brackets, such as in~\cite{Someone:00} or~\cite{Someone:00,Someone:04,Someone:09}. The reference format is the standard IEEE one. We highly recommend you use BibTeX
to generate the reference list.

\section{Conclusions}
Please, submit full-length papers. Submission is fully electronic and automated through the Conference Web Submission System. \underline{Do not} send papers directly by e-mail.


%%%%%%%%%%%%%%%%%%%%%%%%%%%%%%%%%%%%%%%%%%%%%%%%%%%%%%%%%%%%%%%%%%%%%%%%%%%%%
%bibliography here
\bibliography{smc2022bib}

\end{document}
